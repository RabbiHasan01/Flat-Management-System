% Maintain the consistency.
% Maintain a good writing flow. 

\section{Introduction}\label{sec:introduction}
We are exercising this project for learning easy management of flat. The objective of this course is to develop a database application system by applying the theories, methodologies, tools, and technologies we are learned in Database System[CSE-413].

\subsection{Background and Motivation}\label{subsec:bm}
This project is highly important for us to overcome the limitations of the manual data
management system. On account of this problem many companies & organizations lose  their precious time and money. Currently it is really difficult to manage a large amount
of data and search a particular data. Our project can solve the limitations
easily. Because of this process time will be saved and so money will be saved.

\subsection{Problem Statement}\label{subsec:ps}
The Flat Management system is developed for making it easy to find the cottage/mess/flat. This system will contain all information about cottages/messes/flats from where customers can find out their preferable cottages/messes/flats according to their budgets/ choices. Users can perform different kinds
of queries based on their location preferences, budget etc.
\subsection{System Definition}\label{subsec:sd} 
Flat Management system  is used for managing the cottage/mess/flat virtually which can store information about cottages/messes/flats in where customers can easily find out their expected cottages/messes/flats.

A system definition example of a Conference planning system

\textit{``A computerized system used to control the ICCIT conference by registering participants and their payments to organizers using invoicing and other reporting methods. Controlling should be easy to learn, as ICCIT conferences use unpaid and untrained labor."}
\newpage
\noindent
\subsection{System Development Process}\label{subsec:sdp}
We  are following several steps. At first, we collect requirements. Then we
analyze requirements and find some entity types and attributes. With those
entity types and attributes we form an Entity Relationship(ER) diagram. After that we map it onto a relational schema. Then we normalize our database. For developing this system we use PHP as backend language and javascript, HTML, CSS, Jquery for designing frontend.
\begin{comment}


To design a database, one should follow the following steps:
\begin{enumerate}
\item Requirement analysis
	\begin{itemize}
		\item[-] interviewing, documentation, etc .
	\end{itemize}

\item Mapping onto a conceptual model (conceptual design)
     \begin{itemize}
     	\item[-] Entity Relationship(ER) model
     \end{itemize}
\item Mapping onto a data model (logical design)
	\begin{itemize}
     	\item[-] Relational model, object model etc. 
     \end{itemize}
\item Normalization
\item System Architecture
\item Realization and Implementation (physical design)    
    
\end{enumerate}
\end{comment}


\subsection{Organization}  Section~\ref{sec:introduction} gives the overview of the project, Section~\ref{sec:projectmanagement} describes how the project and the resources are managed.
Section~\ref{sec:rga} describes how requirements are gathered and analysed.
Section~\ref{sec:cm} describes how we model our database using Entity Relationship(ER) model, Section~\ref{sec:lm} describes how we convert
our Entity Relationship(ER) model into Relational model, Section~\ref{sec:norm} describes functional dependencies of each relation schema from previous, Section~\ref{sec:lm} and shows that
they are normalized up to 3NF or BCNF, Section~\ref{sec:ncp} describes the overall
architecture of our database system, Section~\ref{sec:sa} describes the whole process of front end and back end of our database system, Section~\ref{sec:imp} describes the success of our product and manual of the system for users, Section~\ref{sec:val} describes the formal proof and disclosure of Flat  Management System, Section~\ref{sec:sd} describes how to install and configure our system so that a non-technical user can use
our system, Finally, the conclusion and the pointers to future work is outlined in section~\ref{sec:cfw}.

\clearpage